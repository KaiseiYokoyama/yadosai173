\documentclass[10pt,a4paper]{jarticle}
 \usepackage{amsmath,amssymb}
 \usepackage{bm}
 \usepackage{pxrubrica}
 \usepackage{color}

\title{けばぶフレンズ 従業員用マニュアル}
\date{\today}

\begin{document}
  \maketitle
  \tableofcontents

  \section{メニュー}
\begin{itemize}
  \item ケバブ
  \begin{description}
    \item[牛] 400yen
    \item[豚] 400yen
  \end{description}
  \item トッピング等 2つで100yen
  \begin{description}
    \item[スイートチリ]
    \item[ニンニク]
    \item[サルサ]
    \item[タルタル]
    \item[ケチャップ]
  \end{description}
  \item 裏メニュー 100yen
  \begin{description}
    \item[デスソース]
    \item[デスソースチャレンジ]
  \end{description}
\end{itemize}

\section{サービス}
\textcolor{red}{重ねがけ可能}
\begin{description}
  \item[宣伝ツイートをRT] 調味料1つ無料
  \item[ハッシュタグ付きツイート] 調味料1つ無料
  \item[10000ツイート超えのユーザーに] 調味料1つ無料
  \item[御輿のクーポンで] 調味料1つ無料
  \item[【裏メニュー(Twitterのみで宣伝)】宣伝アカウントをブロックのユーザーに] デスソースチャレンジ無料
\end{description}
\subsection{デスソースチャレンジ}
\textcolor{red}{Twitterでのみ宣伝}

デスソースのトッピングされたケバブを、水を使わず5分以内に食べ切ればクリア。ケバブ代、デスソースチャレンジ代、その他(注文されていれば)トッピング代を全額キャッシュバックする

\paragraph{補足}
デスソースチャレンジの挑戦は任意。デスソースをトッピングして、チャレンジしないのはあり


\section{各役割ごとの作業手順}
\subsection{調理係}
\subsubsection{パン係(2名)}
電子レンジでパンを加熱し柔らかくした後、水平に包丁を入れてポケットを作ってください
\subsubsection{フライパン係(2名)}
必要に応じてケバブ(牛and豚)を加熱してください。肉の種類に応じてフライパンを変えてください

パン係からパンを受け取り、ポケットに肉を詰めて受付係に渡してください

\subsection{受付係}
お客様に商品を手渡してください。

\subsection{会計係}
会計

\subsection{客引き係}
看板を持って外を周り、宣伝をしてください

\end{document}
